\section{Previous Works}
There exist some approaches which cannot be considered part of the main two classes of Visual Servoing approaches IBVS and PBVS. These approaches differ than the two main visual servoing classes by closing the control loop over projective measures, which are not computed directly from the image space or the 3D pose.

Epipolar geometry can be used to estimate depth, which appears in the interaction matrix of point features\cite{malis20002}. Chesi \textit{et al.} controlled a holonomic mobile robot from a partially calibrated camera using the symmetry of epipolar geometry without point correspondences \cite{chesi}. Benhimane and Malis developed a homography-based approach without reconstructing any 3D parameters \cite{Malis}. Lopez-Nicolas \textit{et al.} designed a homography-based controller which considers the non-holonomic constraints \cite{lopez2006}. These methods use the two-view geometry between the observed and desired views and ignore their relation with the initial view. Epipolar geometry is not well-conditioned if the features are coplanar or the baseline is short, while homography-based approaches require dominant planes.

Recent works introduced using the three-views geometry in the visual servoing loop. The trifocal tensor encapsulates the intrinsic geometry between the three views, and It is analogous for the fundamental matrix in stereo vision systems. The application of trifocal tensor in visual servoing has been neglected until recently. Becerra and Sagues used a simplified trifocal tensor as measurement and estimate and track the pose of a non-holonomic mobile robot with Extended Kalman Filter (EKF) \cite{becerra2009pose}. Lopez-Nicolas \textit{et al.} used the constrained camera motion on a mobile robot and linearized the input-output space for control \cite{lopez2010visual}. This approach provided an analytical interaction matrix, which relates the variations of 9 elements of the trifocal tensor to the motion velocities. Shademan used the trifocal tensor for 6-DOF visual servoing \cite{shademan2010three}. Unlike Lopez-Nicolas, Shademan didn't provide an analytical derivation for the interaction matrix, it was estimated numerically and used in the control law. Shademan argued that using the trifocal tensor to control 6-DOF visual servoing loop was not introduced prior to his work due to the difficulty in linearizing the input-output space in the case of the generalized 6-DOF camera motions.

The work of this thesis is based on the previous approaches introduced by Lopez-Nicolas \cite{lopez2010visual} and Shademan \cite{shademan2010three}. The main contribution of this thesis is to design a generalized 6-DOF visual servoing approach based on the trifocal tensor elements as visual features. The trifocal tensor geometric model is more robust than the two view geometry models as it involves the information given by a third view, and the set of correspondences obtained is more robust to outliers.
