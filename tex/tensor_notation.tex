\section*{Tensor Notation}
Tensors are geometric objects used to represent linear relations between vectors, scalars, and other tensors. A tensor can be represented as a multi-dimensional array of numerical values. The order of a tensor is the dimensionality of the array needed to represent it. Scalars are single numbers and are thus 0th-order tensors. Vectors are 1-dimensional array, 1st-order tensors arranged in a column or row. Matrices are 2-dimensional arrays, 2nd-order tensors arranged as a 2D array of numbers. Similarly, a tensor with three indices may be thought as a 3D array of numbers.

Tensors provide a natural and concise mathematical framework for formulating and solving problems in areas of physics. Tensors express the relationship between vectors, hence they are independent of a particular choice of coordinate system. 

The notation for a tensor is similar to that of a matrix, except that a tensor may have an arbitrary number of indices \textit{e.g.:}$A_{ijk\dotsb}$. In addition, a tensor with rank $r+s$ may be of mixed type $(r,s)$, consisting of $r$ \texttt{contravariant (upper)} indices and $s$ \texttt{covariant (lower)} indices. In tensor notation, a vector $v$ would be written $v_i$, where $i =1,\dotsb,m$, and a matrix is a tensor of type $(1,1)$ would be written as $A^{j}_{i}$.

Tensor notation can provide a very concise way of writing vector and more general identities. For example, the dot product $u.v$ can be simply written as 
$$
u.v = u_{i}v^{i}
$$
where repeated indices are summed over. This is called \texttt{Einstein Summation}. It is a notational convention for simplifying expressions including summations of vectors, matrices and general tensors. The convention can be best illustrated through the following equation
$$
  c^{i}_{k} = a^{i}_{j}b^{j}_{k} = \sum_{j} a_{ij}b_{jk}
$$

Similarly, the cross product can be concisely written as
$$
  (u\times v)_{i} = \epsilon_{ijk} u^{j} v^{k},
$$
where $\epsilon_{ijk}$ is the permutation tensor defined for $r,s,t =1,\dotsb,3$ as follows:
$$
\epsilon_{rst} = \begin{cases}
  0 & \text{ unless } r,s \text{ and } t \text{ are distinct}\\
  +1 & \text{ if } rst \text{ is an even permutation of } 123\\
  -1 & \text{ if } rst \text{ is an odd permutation of } 123
\end{cases}
$$
