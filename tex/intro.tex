\chapter{Introduction} \label{chap:intro}



\section{Problem Definition} \label{sec:problemdefinition}


\section{Objectives}
The purpose of this thesis is to design a visual servoing method for a 6-DOF system based on the trifoal tensor geometric properties. The main objective required is ensure that embedding the trifocal tensor into the visual servoing loop would still allow the servoing system to converge to the desired goal state. This can be achieved by:
\begin{itemize}
  \item Researching the state of the art literature on using the trifocal tensor in visual servoing based systems.
  \item Developing an analytical form to embed the trifocal tensor into the visual servoing loop.
  \item Implementing the method and validate the results in simulations.
  \item Experimenting the method on different robot configurations.
\end{itemize}

\section{Document Organization}
Chapter \ref{chap:background} begins by introducing the basic foundations and concepts that support the rest of this work: Trifocal Geometry, and Visual Servoing. As well as presenting previous works achieved combining these two topics together.
Next, Chapter \ref{chap:vstt} presents the analytical development of the new method and the proposed algorithm to be implemented.
The obtained results are discussed in details in Chapter \ref{chap:results}. Possible improvements and future work are discussed in Chapter \ref{chap:futurework}. Finally, Chapter \ref{chap:conclusion} presents an overall conclusion for the work in hand.
